% ====================================================================
%    ___ ___ ___  __ ___ 
%   / _ (_-</ _ \/ _/ _ \
%   \___/__/\___/\__\___/
%
%   ~ La Empresa de los Programadores Profesionales ~
%
%
%  | http://www.osoco.es
%  |
%  | Edificio Moma Lofts
%  | Planta 3, Loft 18
%  | Ctra. Mostoles-Villaviciosa, Km 0,2
%  | Mostoles, Madrid 28935 España
%
% ====================================================================
%
% Copyleft 2015 
%
% by Jose San Leandro
%
%%%%%%%%%%%%%%%%%%%%%%%%%%%%%%%%%%%%%%%%%%%%%%%%%%%%%%%%%%%%%%%%%%%%%%
\documentclass{beamer}

\usetheme{osoco2012}

\usepackage[english]{babel}
\usepackage[utf8]{inputenc}
\usepackage{minted}
\usepackage{listings}

% Metadata
%%%%%%%%%%%%%%%%%%%%%%%%%%%%%%%%%%%%%%%%%%%%%%%%%%%%%%%%%%%%%%%%%%%%%%

\title{set-square}
\subtitle{Building Docker images your own way}
\author{Jose San Leandro \href{http://twitter.com/rydnr}{@rydnr}}
\institute[OSOCO]{%
  \href{http://www.osoco.es}{OSOCO}
}
\date[11/2015]{November 2015}
\subject{set-square}
\keywords{set-square, phusion, docker, bash}

% Contents
%%%%%%%%%%%%%%%%%%%%%%%%%%%%%%%%%%%%%%%%%%%%%%%%%%%%%%%%%%%%%%%%%%%%%%

\begin{document}

\begin{frame}[plain]
  \titlepage
\end{frame}

\begin{frame}[plain]{Contents}
  \tableofcontents[hideallsubsections]
\end{frame}

\section{Introduction}

\subsection{Motivation}

\begin{frame}{Dockerfiles}
	\begin{columns}[t]
		\column{.5\textwidth}
		\begin{block}{In theory}
			\begin{itemize}
				\item Lorem ipsum.
			\end{itemize}
		\end{block}
		\pause

		\column{.5\textwidth}
		\begin{block}{In practice}
			\begin{itemize}
				\item Lorem ipsum.
			\end{itemize}
		\end{block}

	\end{columns}
\end{frame}

\section{Features}

\begin{frame}{Features}
  \begin{block}{Features}
    \begin{enumerate}
    \item Lorem ipsum
    \end{enumerate}
  \end{block}
\end{frame}

\section{Phusion images}

\begin{frame}{Approach}
  \begin{block}{Approach}
    \begin{enumerate}
    \item Lorem ipsum
    \end{enumerate}
  \end{block}
\end{frame}

\begin{frame}{Built-in modes (I)}
  \begin{block}{Built-in modes}
    \begin{enumerate}
    \item Monit: enabled by default.
    \item Syslog-ng: enabled by default.
    \item rsnapshot: enabled by default (if the image uses volumes)
    \item ssh: disabled by default
    \end{enumerate}
  \end{block}
\end{frame}

\begin{frame}{Built-in modes (II)}
  \begin{block}{Java images}
    \begin{enumerate}
    \item Logstash: enabled by default.
    \end{enumerate}
  \end{block}
\end{frame}

\subsection{base image}

\begin{frame}{Approach}
  \begin{block}{Approach}
    \begin{enumerate}
    \item Lorem ipsum
    \end{enumerate}
  \end{block}
\end{frame}

\subsection{MariaDB/PostgreSQL images}

\begin{frame}{Approach}
  \begin{block}{Approach}
    \begin{enumerate}
    \item Lorem ipsum
    \end{enumerate}
  \end{block}
\end{frame}

\begin{frame}{Bootstrap}
  \begin{block}{Bootstrap}
    \begin{enumerate}
    \item Lorem ipsum
    \end{enumerate}
  \end{block}
\end{frame}

\begin{frame}{Liquibase}
  \begin{block}{Liquibase}
    \begin{enumerate}
    \item Lorem ipsum
    \end{enumerate}
  \end{block}
\end{frame}

\subsection{RabbitMQ/ActiveMQ images}

\begin{frame}{Bootstrap}
  \begin{block}{Bootstrap}
    \begin{enumerate}
    \item Lorem ipsum
    \end{enumerate}
  \end{block}
\end{frame}

\subsection{Web applications}

\begin{frame}{Web applications}
  \begin{block}{Web applications}
    \begin{enumerate}
    \item TT-RSS
    \item Ghost
    \item Getboo
    \item Mediatomb
    \end{enumerate}
  \end{block}
\end{frame}

\end{document}
